%%%%%%%%%%%%%%%%%%%%%%%%%%%%%%%%%%%%%%%%%
% Beamer Presentation
% LaTeX Template
% Version 1.0 (10/11/12)
%
% This template has been downloaded from:
% http://www.LaTeXTemplates.com
%
% License:
% CC BY-NC-SA 3.0 (http://creativecommons.org/licenses/by-nc-sa/3.0/)
%
%%%%%%%%%%%%%%%%%%%%%%%%%%%%%%%%%%%%%%%%%

%----------------------------------------------------------------------------------------
%	PACKAGES AND THEMES
%----------------------------------------------------------------------------------------

\documentclass{beamer}
\mode<presentation> {
\usetheme{Madrid}
}

\usepackage{graphicx} % Allows including images
\usepackage{booktabs} % Allows the use of \toprule, \midrule and \bottomrule in tables
\usepackage[utf8]{inputenc}
\usepackage{lmodern}
\usepackage{outlines}
\usepackage{setspace}
\usepackage{hyperref}
\usepackage{xcolor}
\usepackage{colortbl}
\usepackage{multirow}
\usepackage{tikz}
\usepackage{appendixnumberbeamer}

\setbeamertemplate{button}{\tikz
  \node[
  inner xsep=10pt,
  draw=structure!80,
  fill=structure!50,
  rounded corners=4pt]  {\normalsize\insertbuttontext};}

\setbeamertemplate{itemize items}[square]
\setbeamertemplate{enumerate items}[default]
\setbeamertemplate{navigation symbols}{} % Remove navigation symbols

%----------------------------------------------------------------------------------------
%	TITLE PAGE
%----------------------------------------------------------------------------------------

\title[Modelización de la Estimación de Cáncer]{Modelización Matemática\\ de la Estimación de Incidencia de Cáncer} % The short title appears at the bottom of every slide, the full title is only on the title page

\author[Daniel Redondo-Sánchez]{\textbf{Daniel Redondo-Sánchez}}

\date[11 Julio 2019]{} 

\institute[] % Your institution as it will appear on the bottom of every slide, may be shorthand to save space
{\scriptsize{
\vspace{10pt}
Máster Universitario en Matemáticas\\ Curso 2018/2019\\
\vspace{10pt}
Tutores: \newline \textbf{Francisco Javier Alonso Morales} \newline \textbf{Miguel Rodríguez Barranco}

}
% Your institution for the title page
\vspace{0pt}

% Logos
\begin{columns}
	\begin{column}{0.17\textwidth}
		\begin{figure}
			\centering
			\includegraphics[width=1\textwidth]{logos/mm.pdf}
		\end{figure}
	\end{column}
	\begin{column}{0.5\textwidth}
		\begin{figure}
			\centering
			\includegraphics[width=1\textwidth]{logos/logougr.png}
		\end{figure}
	\end{column}
\end{columns}

}

\begin{document}

\begin{frame}
\titlepage % Print the title page as the first slide
\end{frame}

%----------------------------------------------------------------------------------------
%	PRESENTATION SLIDES
%----------------------------------------------------------------------------------------

%--------------------------------------------------

\begin{frame}\frametitle{}

\Large{\textbf{Índice}}\\[2ex]
\normalsize
\begin{enumerate}
	\item \textbf{Cáncer: Incidencia y Mortalidad} \\[2ex]
	\item Método Razón Incidencia Mortalidad \\[2ex]
	\item Validación del método Razón Incidencia Mortalidad \\[2ex]
	\item Suavizado exponencial \\[2ex]
	\item Estimaciones de la Incidencia de Cáncer en 2018 para Granada \\[2ex]
	\item Líneas abiertas de trabajo
\end{enumerate}

\end{frame}
%--------------------------------------------------

\section{Cáncer: Incidencia y Mortalidad}
\begin{frame}\frametitle{Cáncer: incidencia y mortalidad}

	\begin{block}{Cáncer: un problema mundial de salud pública} 
		\begin{itemize}
			\item El cáncer es un \textbf{conjunto de enfermedades} caracterizadas por una división incontrolada de las células. \\[2ex] 
			\item Es un \textbf{importante problema de salud pública}: 18,1 millones de casos nuevos y 9,6 millones de defunciones en todo el mundo en 2018. \\[2ex]
			\item Para conocer la incidencia en un área geográfica definida es necesario un \textbf{Registro de Cáncer de Población}. 
		\end{itemize}
	\end{block}

	\begin{figure}
		\centering
		\includegraphics[width=0.5\textwidth]{Figuras/mapa_redecan}\\
		\tiny Registros de Cáncer de Población en España
	\end{figure}

\end{frame}

%------------------------------------------------

\begin{frame}\frametitle{Cáncer: incidencia y mortalidad}

	\vspace{-10pt}
	
	\begin{columns}
		\begin{column}{0.33\textwidth}
			\begin{figure}
				\centering
				\includegraphics[width=1\textwidth]{logos/rcg}
			\end{figure}
		\end{column}
		\begin{column}{0.33\textwidth}
		\begin{figure}
			\centering
			\includegraphics[width=1\textwidth]{Figuras/foto_escuela}
		\end{figure}
	\end{column}

		\begin{column}{0.33\textwidth}
			\begin{figure}
				\centering
				\includegraphics[width=1\textwidth]{logos/easp}
			\end{figure}
		\end{column}
	\end{columns}

	\vspace{0pt}

	\begin{block}{Registro de Cáncer de Granada}
		\begin{itemize}
			\item El Registro de Cáncer de Granada se crea en \textbf{1985}, y registra todos los casos de cáncer nuevos diagnosticados en la provincia de Granada. \\[2ex]
			\item En la actualidad cuenta con \textbf{más de 125.000 casos nuevos} de cáncer. \\[2ex] 
			\item Está integrado en las instituciones sobre cáncer de más alto prestigio internacional (IACR, ENCR, IARC), y sus datos son de \textbf{alta calidad}, estando acreditados por la Organización Mundial de la Salud. \\[2ex] 
		\end{itemize}
	\end{block}

\end{frame}

%--------------------------------------------------
\begin{frame}\frametitle{Cáncer: incidencia y mortalidad}

	\begin{block}{Indicadores epidemiológicos}
		Para medir el impacto del cáncer en una población se utilizan los siguientes indicadores:\\
		\begin{enumerate}
			\item \textbf{Incidencia. Riesgo de presentar cáncer.}\\[2ex]
			\item \textbf{Mortalidad. Riesgo de morir por cáncer.}\\[2ex]
			\item Prevalencia. Carga asistencial de la enfermedad.\\[2ex]
			\item Supervivencia. Historia natural de la enfermedad y efectividad del tratamiento. \\[2ex]
		\end{enumerate}
	\end{block}

\end{frame}

%--------------------------------------------------

\begin{frame}\frametitle{Cáncer: incidencia y mortalidad}

	\begin{block}{Indicadores para la incidencia}
		
		El indicador más básico es el \textbf{número de casos nuevos de cáncer}.\\[2ex]
		
		Para tener en cuenta el tamaño de la población, o su distribución por edad, se calculan otros indicadores como \textbf{tasas brutas}, \textbf{tasas específicas por edad}, o \textbf{tasas estandarizadas por edad}.
	\end{block}

	\begin{block}{Indicadores para la mortalidad}
		Se pueden calcular las mismas tasas que para la incidencia, cambiando el número de casos nuevos por el número de \textbf{defunciones por cáncer}.
	\end{block}
\end{frame}

%--------------------------------------------------

\begin{frame}\frametitle{Cáncer: incidencia y mortalidad}
	\begin{block}{Razón incidencia mortalidad}
		La \textbf{razón incidencia mortalidad} (RIM) se define como el cociente entre el número de casos nuevos de cáncer y el número de defunciones por cáncer.
		
		\vspace{0pt}
		$$\text{Razón incidencia mortalidad} = \frac{\text{Casos nuevos de cáncer}}{\text{Defunciones por cáncer}}$$ \\[1ex]
		
		Por lo general, la razón incidencia-mortalidad oscila entre 0 y 1, aunque puede ser mayor de 1 en cánceres muy letales (pulmón, páncreas, ...).\\[2ex]
		
		A veces es malinterpretada como supervivencia, aunque la mortalidad se refiere a un grupo distinto de pacientes que la incidencia.
	\end{block}

\end{frame}

%--------------------------------------------------
\begin{frame}\frametitle{}

\Large{\textbf{Índice}}\\[2ex]
\normalsize
\begin{enumerate}
	\item Cáncer: Incidencia y Mortalidad \\[2ex]
	\item \textbf{Método Razón Incidencia Mortalidad} \\[2ex]
	\item Validación del método Razón Incidencia Mortalidad \\[2ex]
	\item Suavizado exponencial \\[2ex]
	\item  Estimaciones de la Incidencia de Cáncer en 2018 para Granada \\[2ex]
	\item Líneas abiertas de trabajo
\end{enumerate}

\end{frame}

%--------------------------------------------------
\section{Método Razón Incidencia Mortalidad}
\begin{frame}\frametitle{Método Razón Incidencia Mortalidad}

	\begin{block}{Estimaciones de la incidencia}
		Las estimaciones de la incidencia de cáncer son \textbf{necesarias para conocer la carga actual de enfermedad} en una población, ya que los Registros de Cáncer tienen un retraso de unos 4-5 años.\\[2ex]
		
	El método razón incidencia mortalidad (\textbf{método RIM}) es el más adecuado al contexto de España, donde se dispone de varios Registros de Cáncer de Población regionales, y un Registro de Mortalidad a nivel nacional.
	\end{block}
\vspace{-5pt}
	\begin{figure}
		\centering
		\includegraphics[width=0.8\textwidth]{Figuras/eleccion_metodo}
	\end{figure}
\vspace{-15pt}
\centering {\tiny An assessment of GLOBOCAN methods for deriving national estimates of cancer incidence. Sebastien Antoni et al. 2016}


\end{frame}

%--------------------------------------------------

\begin{frame}\frametitle{Método Razón Incidencia Mortalidad}

		\begin{block}{Fase 1. Proyección de la mortalidad}
			Se proyecta la mortalidad para el año objetivo usando el método \textbf{NORDPRED}:
			
			$$R_{ap} = {(A_a + D\cdot p + P_p + C_c)}^5$$ donde:
			
			\begin{itemize}
				\item $R_{ap}$ es la mortalidad para un grupo de edad ($a$) y un periodo ($p$) específicos
				\item $A_a$ es la componente de la edad para un grupo de edad concreto $a$
				\item $D$ es el parámetro de deriva que refleja la componente lineal de la tendencia
				\item $P_p$ es la componente no lineal de un periodo específico $p$
				\item $C_c$ es la componente no lineal de una cohorte específica $c$
			\end{itemize}
		\end{block}


\end{frame}

%--------------------------------------------------

\begin{frame}\frametitle{Método Razón Incidencia Mortalidad}

	\begin{block}{Fase 2. Modelado RIM}
		Se modela la RIM usando un \textbf{modelo lineal generalizado mixto} (GLMM)
		$$log(N) =  log(D) + \beta_0 + \beta_1 \cdot y + \beta_2 \cdot y^2  + \sum_{i = 1}^{4} \alpha_i \cdot a_i $$ donde:
		\begin{itemize}
			\item $N$ = número de casos incidentes 
			\item $D$ = número de defunciones 
			\item $a_i$ = Grupos de edad ($a_i$, con $i = 1, \hdots, 4$)
			\item $y$ = Año de diagnóstico
		\end{itemize}
	
	\vspace{10pt}
	Equivalentemente, $RIM = \mathrm e^{\beta_0 + \beta_1 \cdot y + \beta_2 \cdot y^2 + \sum_{i = 1}^{4} \alpha_i \cdot a_i} $

	\end{block}
\end{frame}

%--------------------------------------------------

\begin{frame}\frametitle{Método Razón Incidencia Mortalidad}
	\begin{block}{Fase 2. Modelado RIM}
		
		La estimación de los parámetros se realiza mediante \textbf{cadenas de Markov Montecarlo (MCMC)}.\\[2ex]
		
		Se utilizaron los paquetes de R \texttt{BRugs} v.0.9-0 y \texttt{R2WinBUGS} v.2.1-21, y el software \textbf{OpenBUGS} v.3.2.3., la versión \textit{open source} de WinBugs.\\[2ex]
		
		Se realizan 3 cadenas de simulaciones con 10.000 ciclos de calentamiento y 40.000 ciclos para el análisis. 1 de cada 5 ciclos fue seleccionado para obtener \textbf{24.000 estimaciones de los parámetros}.\\[2ex]
		
	Se pueden considerar los percentiles 2,5 y 97,5 para obtener un \textbf{intervalo de credibilidad al 95\%}.\\
		
	\end{block}

\end{frame}



%--------------------------------------------------

\begin{frame}\frametitle{Método Razón Incidencia Mortalidad}

	\begin{block}{Fase 2. Modelado RIM. Escenarios}
		\begin{itemize}
			\item \textbf{C1}. Se asume que la RIM sigue una tendencia constante, e igual a la RIM del \textbf{último año} del que se conoce la incidencia.\\[2ex]
			\item \textbf{L}. Se asume que la RIM sigue una tendencia \textbf{lineal}.\\[2ex]
			\item \textbf{Q}. Se asume que la RIM sigue una tendencia \textbf{cuadrática}.\\[2ex]
			\pause
			\item \textbf{C3}. Se asume que la RIM sigue una tendencia constante, e igual a la media de las RIM de los \textbf{últimos 3 años} de los que se conoce la incidencia.\\[2ex]
			\item \textbf{C5}. Se asume que la RIM sigue una tendencia constante, e igual a la media de las RIM de los \textbf{últimos 5 años} de los que se conoce la incidencia.\\[2ex]
		\end{itemize}
	\end{block}

\end{frame}

%--------------------------------------------------

\begin{frame}\frametitle{Método Razón Incidencia Mortalidad}

	\begin{block}{Fase 3. Cálculo incidencia}
		Para cada sexo, grupo de edad y localización anatómica:
		\begin{itemize}
			\item En la fase 1 se ha obtenido la mortalidad del año objetivo. \\[2ex]
			\item En la fase 2 se ha modelizado la RIM del año objetivo para cada escenario.\\[2ex]
		\end{itemize}		
		
		Podemos calcular el \textbf{número de casos incidentes, para cada sexo, grupo de edad, localización anatómica y escenario}, usando:\\
		
		$$\text{RIM} = \dfrac{\text{Casos nuevos de cáncer}}{\text{Defunciones por cáncer}} = \dfrac{N}{D} \Rightarrow N  = \text{RIM} \cdot D $$
		
		Posteriormente, se pueden calcular otros indicadores (tasas brutas, tasas estandarizadas por edad, ...)
	\end{block}

\end{frame}

%--------------------------------------------------


\begin{frame}\frametitle{}

\Large{\textbf{Índice}}\\[2ex]
\normalsize
\begin{enumerate}
	\item Cáncer: Incidencia y Mortalidad \\[2ex]
	\item Método Razón Incidencia Mortalidad \\[2ex]
	\item \textbf{Validación del método Razón Incidencia Mortalidad} \\[2ex]
	\item Suavizado exponencial \\[2ex]
	\item  Estimaciones de la Incidencia de Cáncer en 2018 para Granada \\[2ex]
	\item Líneas abiertas de trabajo
\end{enumerate}

\end{frame}

%--------------------------------------------------
\section{Validación del método RIM}
\begin{frame}\frametitle{Validación del método RIM}

\vspace{-30pt}
\begin{block}{Validación método RIM}
	El método RIM se aplica para cada año entre \textbf{2004} y \textbf{2013}. Posteriormente, \textbf{se compara la estimación obtenida con el valor real observado}.\\
\end{block}

\begin{figure}
	\centering
	\includegraphics[width=1\textwidth]{Figuras/Figura5_primeralinea}
\end{figure}

\vspace{30pt}
\begin{figure}
	\centering
	\includegraphics[width=1\textwidth]{Figuras/Figura5_leyenda}
\end{figure}

\pause

\vspace{-133pt}
\begin{figure}
	\centering
	\includegraphics[width=1\textwidth]{Figuras/Figura5_resto}
\end{figure}

\end{frame}

%--------------------------------------------------

\begin{frame}\frametitle{Validación del método RIM}

\begin{block}{Fuentes de información}
	\begin{itemize}
		\item \textbf{Incidencia}: \textbf{Registro de Cáncer de Granada}. Casos incidentes, 1985-2013, por año, sexo, grupo de edad y localización anatómica.\\[2ex]
		\item \textbf{Mortalidad}: \textbf{Ministerio de Sanidad, Consumo y Bienestar Social}. Defunciones por cáncer, 1982-2013, por año, sexo, grupo de edad y causa de muerte.\\[2ex]
		\item \textbf{Población}: \textbf{Instituto Nacional de Estadística}. Población a 1 de Julio, 1982-2013, por año, sexo y grupo de edad.\\[2ex]
	\end{itemize}
	Se utiliza Microsoft Access (\textbf{VBA} + \textbf{SQL}) y \textbf{R} para procesar la información y crear los ficheros necesarios.
\end{block}

\end{frame}

%--------------------------------------------------

\begin{frame}\frametitle{Validación del método RIM}
	\begin{block}{Metodología}
		Se analizan las siguientes localizaciones anatómicas (códigos CIE-10):
		\begin{table}[]
			\begin{tabular}{ll}
				\textbf{Hombres} & \textbf{Mujeres} \\ \hline
				Colon (C18) & Colon (C18) \\ [1ex]
				Recto (C19-20) & Recto (C19-20)  \\ [1ex]
				Pulmón (C33-34) & Pulmón (C33-34) \\ [1ex]
				Próstata (C61) & Mama (C50) \\ [1ex]
				Vejiga (C67, D09, D41) & Cuerpo uterino (C54) \\ [1ex]
				Estómago (C16) & Ovario (C56)\\ [1ex]
				Otras & Otras \\[1ex]
			\end{tabular}
		\end{table}
	\end{block}
\end{frame}



%--------------------------------------------------

\begin{frame}\frametitle{Validación del método RIM}

\begin{block}{Indicadores de la bondad de ajuste}
	\textbf{Desviación relativa} (DR) para un escenario $s$:
	
	$$ \text{DR} = 100 \cdot \left\lvert \dfrac{E(s) - O}{O}\right\rvert$$\\[2ex]
	
	\textbf{Error absoluto porcentual medio} (MAPE, por sus siglas en inglés: \textit{Mean Absolute Percentage Error}) como:
	
	$$\text{MAPE}(s) =  \dfrac{100}{10} \sum_{i = 2004}^{2013} \left\lvert \dfrac{E_i(s)-O_i}{O_i} \right\lvert$$\\[2ex]
	
	{\centering
	$s_1$ es \textbf{mejor escenario} que $s_2$ $\iff$ $\text{MAPE}(s_1) < \text{MAPE}(s_2)$\\[2ex]}
	
	El mejor escenario será entonces aquel que obtenga el menor MAPE.
\end{block}

\end{frame}

%--------------------------------------------------

\begin{frame}\frametitle{Validación del método RIM}

\begin{block}{Cáncer de pulmón en hombres, provincia de Granada, 2004-2013}
	\small  Número de casos observados y estimados por dos escenarios (C3 y L).
	\vspace{-10pt}
	\begin{center}
		\begin{tikzpicture}
		\node[align=center, inner sep=0] (image) at (0,0) {\includegraphics[width=1\textheight]{Figuras/comparacion_escenarios.pdf}}; 
		\end{tikzpicture}
	\end{center}
\end{block}

\end{frame}

%--------------------------------------------------

\begin{frame}\frametitle{Validación del método RIM}

	\begin{block}{Cáncer de pulmón en hombres, provincia de Granada, 2004-2013}
		\small  Número de casos observados y estimados por dos escenarios (C3 y L).
		\vspace{-10pt}
		\begin{center}
			\begin{tikzpicture}
				\node[align=center, inner sep=0] (image) at (0,0) {\includegraphics[width=1\textheight]{Figuras/comparacion_escenarios.pdf}};
				\node[align=left, font={\footnotesize}] at (0,-0.33) {\textbf{MAPE(C3) = 4,5}}; % at (image.center)
				\node[align=left, font={\footnotesize}] at (0,-0.8) {MAPE(L)  = 5,5}; % at (image.center)
			\end{tikzpicture}
		\end{center}
	\end{block}
	
\end{frame}

%--------------------------------------------------

\begin{frame}\frametitle{Validación del método RIM}

\vspace{-10pt}
\begin{table}[H]
	\small
	\centering 
	\begin{tabular}{llrrrrl}
		\hline
		\textbf{Sexo} & \begin{tabular}[l]{@{}l@{}}\textbf{Localización}\\ \textbf{anatómica}\end{tabular} & \textbf{Obs.} & \textbf{Est.} & \textbf{DR} & \textbf{MAPE} & \begin{tabular}[l]{@{}l@{}}\textbf{Mejor}\\ \textbf{escenario}\end{tabular}\\ \hline
		\multirow{8}{*}{Hombres}	 & Colon & 1.963 & 1.888 & 3,8 & 8,6 & C3\\
		& Recto & 1.084 & 985 & 9,1 & 16,8 & C3\\
		& Pulmón & 3.391 & 3.339 & 1,5 & 4,5 & C3\\
		& Próstata & 4.424 & 4.936 & 11,6 & 27,4 & L\\
		& Vejiga  & 2.635 & 2.594 & 1,6 & 8,5 & C1\\
		& Estómago & 801 & 760 & 5,1 & 14,1 & C5\\
		& Otros & 8.899 & 8.624 & 3,1 & 3,5 & C3\\
		& Total & 23.197 & 23.126 & 0,3 & 6,3 & \\  \hline
				\multirow{8}{*}{Mujeres}	 & Colon & 1.467 & 1.413 & 3,7 & 10,0 & C5\\
		& Recto & 669 & 615 & 8,1 & 19,2 & C3\\
		& Pulmón & 557 & 487 & 12,6 & 18,3 & L\\
		& Mama & 4.220 & 4.082 & 3,3 & 14,8 & C3\\
		& Cuerpo uterino & 1.134 & 1.262 & 11,3 & 17,3 & C5\\
		& Ovario & 615 & 703 & 14,3 & 18,9 & L\\
		& Otros & 7.989 & 8.012 & 0,3 & 8,7 & L\\
		& Total & 16.651 & 16.574 & 0,5 & 3,8&\\ \hline
	\end{tabular}
\end{table}

\end{frame}

%--------------------------------------------------

\begin{frame}\frametitle{Validación del método RIM}

Colon, mujeres, escenario C5. MAPE = 10,0
\begin{figure}
	\centering
	\includegraphics[width=.85\textwidth]{Figuras/colon_mujeres}
\end{figure}

\end{frame}

%--------------------------------------------------

\begin{frame}\frametitle{Validación del método RIM}

Pulmón, hombres, escenario C3. MAPE = 4,5
\begin{figure}
	\centering
	\includegraphics[width=.85\textwidth]{Figuras/pulmon_hombres}
\end{figure}

\end{frame}

%--------------------------------------------------

\begin{frame}\frametitle{Validación del método RIM}

	\begin{block}{Conclusiones}
		\begin{itemize}
			\item El MAPE \textbf{proporciona el mejor escenario} para cada localización anatómica y sexo, así como \textbf{una medida cuantitativa de la bondad de ajuste} de esa estimación. Se mejoran así métodos subjetivos implantados en la actualidad.\\[2ex]
			\item El escenario Q no es adecuado (en Granada), pese a usarse en estimaciones nacionales.\\[2ex]
			\item El método RIM es menos preciso con pocos casos incidentes o defunciones, o ante \textbf{cambios repentinos en la evolución de la RIM} (programas de cribado).
		\end{itemize}
	\end{block}

\end{frame}

%--------------------------------------------------

\begin{frame}\frametitle{Validación del método RIM}

\begin{block}{Relación DR - Número medio anual de casos observados}
	\begin{figure}
		\centering
		\includegraphics[width=.9\textwidth]{Figuras/figura8.png}
	\end{figure}
\end{block}

\end{frame}

% hombres, mayores de 50 años -> PSA, desde 1997.
% PSA =  antígeno prostático específico
% Biopsia transrectal tras PSA positivo
% Muchos cánceres latentes
% Sobrediagnóstico y adelanto diagnóstico
% Gran aumento en incidencia +6.3% 1985-2013
% Mortalidad estable (-0,4%)

%--------------------------------------------------

\begin{frame}\frametitle{Validación del método RIM}

Próstata, escenario L. MAPE = 27,4
\begin{figure}
	\centering
	\includegraphics[width=0.85\textwidth]{Figuras/prostata_hombres}
\end{figure}

\end{frame}


%--------------------------------------------------


\begin{frame}\frametitle{Validación del método RIM}
\vspace{-10pt}
\begin{figure}
	\centering
	\includegraphics[width=0.85\textwidth]{Figuras/prostata_menos5}
\end{figure}

\end{frame}

%--------------------------------------------------

\begin{frame}\frametitle{Validación del método RIM}
\vspace{-10pt}

\begin{figure}
\centering
\includegraphics[width=0.85\textwidth]{Figuras/prostata}
\end{figure}

\end{frame}

% -------------------------------------------------

\begin{frame}\frametitle{Validación del método RIM}

\begin{block}{Richard J. Ablin, descubridor del PSA}
	
	\begin{columns}
		\begin{column}{0.3\textwidth}
			\begin{figure}
				\centering
				\includegraphics[width=1\textwidth]{Figuras/richard-ablin}
			\end{figure}
		\end{column}
		\begin{column}{0.625\textwidth}	
			\textit{'I never dreamed that my discovery four decades ago would lead to such a profit-driven public health disaster.\\
				The medical community must confront reality and stop the inappropriate use of PSA screening. Doing so would save billions of dollars and rescue millions of men from unnecessary, debilitating treatments.'}\\[2ex]
			
			- Richard J. Ablin,  \textit{The New York Times}
		\end{column}
	\end{columns}
	
\end{block}

\end{frame}


%--------------------------------------------------

\begin{frame}\frametitle{Validación del método RIM}

\vspace{-10pt}
\begin{columns}
	\begin{column}{0.5\textwidth}
		\begin{figure}
			\centering
			\includegraphics[width=1\textwidth]{Figuras/articulo.png}
		\end{figure}
	\end{column}
	\begin{column}{0.5\textwidth}
		\begin{figure}
			\centering
			\includegraphics[width=1\textwidth]{Figuras/dani_canada}
		\end{figure}
	\end{column}
\end{columns}

\begin{block}{Difusión científica}
	Artículo enviado a \texttt{BMC Population Health Metrics} (revista indexada JCR, Q1 en 2018 en ``Salud Pública, ambiental y ocupacional").\\ [2ex]
	
	Presentado en cinco congresos científicos nacionales e internacionales (Canadá, Italia, Portugal, Valencia, Granada).
\end{block}

\end{frame}

%--------------------------------------------------

\begin{frame}\frametitle{}

	\Large{\textbf{Índice}}\\[2ex]
	\normalsize
	\begin{enumerate}
		\item Cáncer: Incidencia y Mortalidad \\[2ex]
		\item Método Razón Incidencia Mortalidad \\[2ex]
		\item Validación del método Razón Incidencia Mortalidad \\[2ex]
		\item \textbf{Suavizado exponencial} \\[2ex]
		\item  Estimaciones de la Incidencia de Cáncer en 2018 para Granada \\[2ex]
		\item Líneas abiertas de trabajo
	\end{enumerate}

\end{frame}

%--------------------------------------------------
\section{Suavizado exponencial}
\begin{frame}\frametitle{Suavizado exponencial}

	\begin{block}{Teoría de suavizado exponencial}
		Sea $\{x_1, x_2, \dots, x_n\}$ la serie temporal de casos incidentes.\\[2ex]
		
		$\{x_t\}$ se puede estimar usando suavizado exponencial como $\{a_t\}$, con:\\[2ex]
		
		\begin{math}
		\left\{
		\begin{array}{l}
		a_1 = x_1\\[2ex]
		a_t = \alpha x_t + (1-\alpha) a_{t-1} = \ldots = \sum_{i=1}^t \alpha (1-\alpha)^{t-i} x_i
		\end{array}
		\right.
		\end{math}\\[2ex]
		
siendo $\alpha \in (0, 1]$ el coeficiente de suavizado.\\[2ex]
		
		\textbf{Se da más importancia a las observaciones más recientes.}\\
	
	\end{block}

\end{frame}

%--------------------------------------------------

\begin{frame}\frametitle{Suavizado exponencial}
	
	\small
	\begin{block}{Cálculo del mejor coeficiente de suavizado}
		
		Se desarrolla un script de R en el que:
		\begin{enumerate}
			\item Se proyectan las defunciones usando \textbf{NORDPRED}.
			\item Se realiza una \textbf{partición del intervalo} $(0, 1]$ en 20 valores que tomará $\alpha$.\\[2ex] %(debido al alto coste computacional: 14 localizaciones $\times$ 10 años $\times$ 20 valores de $\alpha$ = 2.800 estimaciones).\\[2ex]
			\item Para cada $\alpha$, sexo, localización anatómica y año entre 2004 y 2013, \textbf{se modeliza la RIM usando suavizado exponencial}, desde el año 1985 hasta 5 años antes del año objetivo.
			\item Se calcula el \textbf{número de casos incidentes} multiplicando RIM modelada por defunciones proyectadas.\\
			\item Se selecciona el coeficiente de suavizado que \textbf{minimiza el MAPE}.
		\end{enumerate}

	\end{block}

\end{frame}

%--------------------------------------------------

\begin{frame}\frametitle{Suavizado exponencial}
	\begin{columns}
		\begin{column}{0.5\textwidth}
		\centering	Cáncer de recto en hombres
			\begin{figure}
				\centering
				\includegraphics[width=1\textwidth]{Figuras/115-20}
			\end{figure}
		\end{column}
		\begin{column}{0.5\textwidth}
		\centering	Cáncer de recto en mujeres
		\begin{figure}
			\centering
			\includegraphics[width=1\textwidth]{Figuras/615-20}
		\end{figure}
	\end{column}
	\end{columns}


\end{frame}
%--------------------------------------------------

\begin{frame}\frametitle{Suavizado exponencial}
\begin{columns}
	\begin{column}{0.5\textwidth}
		\centering Cáncer de pulmón en hombres\\ (n = 20)
		\begin{figure}
			\centering
			\includegraphics[width=1\textwidth]{Figuras/116-20}
		\end{figure}
	\end{column}
	\begin{column}{0.5\textwidth}
		\centering Cáncer de pulmón en hombres\\ (n = 100)
			\begin{figure}
				\centering
				\includegraphics[width=1\textwidth]{Figuras/116-100}
			\end{figure}
		\end{column}
\end{columns}


\end{frame}


%--------------------------------------------------

\begin{frame}\frametitle{Suavizado exponencial}

Validación de la estimación de la incidencia. Granada, 2004-2013.

\begin{table}[H]
	\centering 
	\small
	\begin{tabular}{llrcrc}
		& & \multicolumn{2}{c}{\textbf{Método RIM}} & \multicolumn{2}{c}{\textbf{Suavizado exponencial}}  \\ \hline
		Sexo&\begin{tabular}[l]{@{}l@{}}Localización\\ anatómica\end{tabular}&MAPE&\begin{tabular}[c]{@{}c@{}}Mejor\\ escenario\end{tabular}&MAPE& Mejor $\alpha$\\ \hline
		\multirow{8}{*}{Hombres}	&Colon&8,6&C3&\textbf{7,3}&\textbf{0,1}\\
		&Recto&16,8&C3&\textbf{12,5}&\textbf{0,4}\\
		&Pulmón&\textbf{4,5}&\textbf{C3}&4,6&0,35\\
		&Próstata&\textbf{27,4}&\textbf{L}&28,9&0,8\\
		&Vejiga &8,5&C1&\textbf{5,9}&\textbf{0,6}\\
		&Estómago&14,1&C5&\textbf{13,3}&\textbf{0,15}\\
		&Otros&\textbf{3,5}&\textbf{C3}&3,9&0,4\\ \hline
				\multirow{8}{*}{Mujeres}	&Colon&\textbf{10,0}&\textbf{C5}&10,5&1\\
		&Recto&19,2&C3&\textbf{14,4}&\textbf{0,25}\\
		&Pulmón&18,3&L&\textbf{15,2}&\textbf{0,5}\\
		&Mama&14,8&C3&\textbf{13,0}&\textbf{0,2}\\
		&Cuerpo uterino&\textbf{17,3}&\textbf{C5}&35,8&0,55\\
		&Ovario&\textbf{18,9}&\textbf{L}&26,2&0,85\\
		&Otros&\textbf{8,7}&\textbf{L}&11,8&1\\ \hline
	\end{tabular}
\end{table}
\end{frame}


%--------------------------------------------------

\begin{frame}\frametitle{}

	\Large{\textbf{Índice}}\\[2ex]
	\normalsize
	\begin{enumerate}
		\item Cáncer: Incidencia y Mortalidad \\[2ex]
		\item Método Razón Incidencia Mortalidad \\[2ex]
		\item Validación del método Razón Incidencia Mortalidad \\[2ex]
		\item Suavizado exponencial \\[2ex]
		\item \textbf{ Estimaciones de la Incidencia de Cáncer en 2018 para Granada} \\[2ex]
		\item Líneas abiertas de trabajo
	\end{enumerate}

\end{frame}

%--------------------------------------------------
\section{ Estimaciones de la Incidencia de Cáncer en 2018 para Granada}
\begin{frame}\frametitle{Estimaciones de la Incidencia de Cáncer. Granada, 2018}

	\begin{block}{Estimaciones de la Incidencia en 2018 para Granada}
		Las estimaciones en 2018 para la provincia de Granada son útiles para mejorar la planificación de la atención sanitaria y la vigilancia de la enfermedad.
	\end{block}

	\begin{block}{Fuentes de información}
		\begin{itemize}
			\item \textbf{Incidencia}: \textbf{Registro de Cáncer de Granada}. Casos incidentes de cáncer de Granada, periodo 1985-2013.
			\item \textbf{Mortalidad}: \textbf{Ministerio de Sanidad, Consumo y Bienestar Social}. Defunciones por cáncer en la provincia de Granada, periodo 1985-2015.
			\item \textbf{Población}: \textbf{Instituto Nacional de Estadística}. Población a 1 de Julio, periodo 1985-2015. Proyección de población intercensal para 2018.
		\end{itemize}
	\end{block}

\end{frame}

%--------------------------------------------------
\begin{frame}\frametitle{Estimaciones de la Incidencia de Cáncer. Granada, 2018}
	
	\begin{block}{Estimaciones con método RIM}
		\begin{table}[H]
			\footnotesize
			\centering 
			\begin{tabular}{llrrl}
				\hline
				\textbf{Sexo}&\begin{tabular}[l]{@{}l@{}}\textbf{Localización}\\ \textbf{anatómica}\end{tabular}&\textbf{N}& \textbf{IC 95}\% & \textbf{Esc}\\ \hline
				\multirow{8}{*}{Hombres}	&Colon&240 & (206, 275)&C3\\
				&Recto&117 & (94, 142) &C3\\
				&Pulmón&342 & (302, 384)&C3\\
				&Próstata&752 & (672, 839)&L\\
				&Vejiga &356 & (310, 405)&C1\\
				&Estómago&64 & (48, 83)&C5\\
				&Otros&917 & (850, 985)&C3\\
				&Total&2.788 & (2.482, 3.113)&\\ \hline
				\multirow{8}{*}{Mujeres}	&Colon&148 & (123, 175)&C5\\
				&Recto&44 & (31, 60)&C3\\
				&Pulmón&78 & (55, 106)&L\\
				&Mama&565 & (513, 620)&C3\\
				&Cuerpo uterino&133 & (108, 159)&C5\\
				&Ovario&47 & (31, 67)&L\\
				&Otros&898 & (817, 982)&L\\
				&Total  &1.913 & (1.678, 2.169)&\\ \hline
			\end{tabular}
		\end{table}
	\end{block}
\end{frame}

%--------------------------------------------------

\begin{frame}\frametitle{Estimaciones de la Incidencia de Cáncer. Granada, 2018}
	\begin{block}{Estimaciones con método RIM + suavizado exponencial}
		\begin{table}[H]
			\footnotesize
			\centering 
			\begin{tabular}{llrrl}
				\hline
				\textbf{Sexo}&\begin{tabular}[l]{@{}l@{}}\textbf{Localización}\\ \textbf{anatómica}\end{tabular}&\textbf{N}&\textbf{IC 95\%}&\textbf{Escenario}\\ \hline
				\multirow{8}{*}{Hombres}	&Colon&237 & (158, 316)&$SE_{\alpha = 0,1}$\\
				&Recto&117 & (38, 195)&$SE_{\alpha = 0,4}$\\
				&Pulmón&342 & (302, 384)&C3\\
				&Próstata&752 & (672, 839)&L\\
				&Vejiga &350 & (209, 492)&$SE_{\alpha = 0,6}$\\
				&Estómago&66 & (44, 88)&$SE_{\alpha = 0,15}$\\
				&Otros&917 & (850, 985)&C3\\
				&Total &2.781 & & \\ \hline
				\multirow{8}{*}{Mujeres}	&Colon&148 & (123, 175)&C5\\
				&Recto&51 & (15, 88)&$SE_{\alpha = 0,25}$\\
				&Pulmón&72 & (16, 129)&$SE_{\alpha = 0,5}$\\
				&Mama&538 & (402, 674)&$SE_{\alpha = 0,2}$\\
				&Cuerpo uterino&133 & (108, 159)&C5\\
				&Ovario&47 & (31, 67)&L\\
				&Otros&898 & (817, 982)&L\\
				&Total &1.887 & &\\ \hline
			\end{tabular}
		\end{table}
	\end{block}
\end{frame}

%--------------------------------------------------

\begin{frame}\frametitle{Estimaciones de la Incidencia de Cáncer. Granada, 2018}
	\begin{block}{Estimaciones con método RIM + suavizado exponencial}
		\begin{itemize}
			\item \textbf{4.668 nuevos casos de cáncer }en 2018 para la provincia de Granada.
			\item\textbf{ 59,6\%} de los casos son diagnosticados \textbf{en hombres}.
			\item Cánceres más frecuentes en hombres: \textbf{próstata} (752 casos), \textbf{vejiga} (350 casos) y \textbf{pulmón} (342 casos).
			\item Cánceres más frecuentes en mujeres: \textbf{mama} (538 casos), \textbf{colon} (148 casos) y \textbf{cuerpo uterino} (133 casos).
		\end{itemize}
	\end{block} \pause
	\begin{block}{Notas sobre el método}
		\begin{itemize}
			\item Las estimaciones para los cánceres de próstata, mama y colon deben ser tomadas con cautela.
			\item El suavizado exponencial da estimaciones similares ($\pm 7$ casos) al método RIM salvo para cáncer de mama (27 casos menos).
		\end{itemize}
	\end{block}
\end{frame}

%--------------------------------------------------

\begin{frame}\frametitle{Estimaciones de la Incidencia de Cáncer. Granada, 2018}
\vspace{-10pt}
		\begin{figure}
			\centering
			\includegraphics[width=1\textwidth]{Figuras/pulmon}
		\end{figure}


\end{frame}
%--------------------------------------------------

\begin{frame}\frametitle{Estimaciones de la Incidencia de Cáncer. Granada, 2018}
\vspace{-10pt}

		\begin{figure}
			\centering
			\includegraphics[width=1\textwidth]{Figuras/mama}
		\end{figure}

\end{frame}

%--------------------------------------------------

\begin{frame}\frametitle{}

\Large{\textbf{Índice}}\\[2ex]
\normalsize
\begin{enumerate}
	\item Cáncer: Incidencia y Mortalidad \\[2ex]
	\item Método Razón Incidencia Mortalidad \\[2ex]
	\item Validación del método Razón Incidencia Mortalidad \\[2ex]
	\item Suavizado exponencial \\[2ex]
	\item Estimaciones de la Incidencia de Cáncer en 2018 para Granada \\[2ex]
	\item \textbf{Líneas abiertas de trabajo}
\end{enumerate}

\end{frame}

%--------------------------------------------------
\section{Líneas abiertas de trabajo}
\begin{frame}\frametitle{Líneas abiertas de trabajo}

	\begin{block}{Líneas abiertas de trabajo}
		\begin{itemize}
			\item \textbf{Proyectos colaborativos multicéntricos}, para tener en cuenta las desigualdades entre regiones, y ganar potencia estadística. \\[2ex]
			\item \textbf{Método de Brown}, y otras técnicas de análisis de series temporales. \\[2ex]
			\item \textbf{Actualización periódica de las estimaciones}, para tener datos actualizados en todo momento, de utilidad para la planificación sanitaria.\\[2ex]
		\end{itemize}
	\end{block}

\end{frame}

%--------------------------------------------------

\begin{frame}\frametitle{Referencias bibliográficas}

\vspace{-10pt}
\begin{columns}
	\begin{column}{0.5\textwidth}
		\begin{figure}
				\centering
				\includegraphics[width=1\textwidth]{Figuras/redecan2015}
			\end{figure}
			\begin{figure}
				\raggedright
				\includegraphics[width=0.8\textwidth]{Figuras/redecan}
			\end{figure}
	\end{column}
	\begin{column}{0.5\textwidth}
		\begin{figure}
			\centering
			\includegraphics[width=0.85\textwidth]{Figuras/portada_cowp}
		\end{figure}
	\end{column}

\end{columns}

\end{frame}


%------------------------------------------------

\begin{frame}
	
	\Huge{\centerline{Gracias por su atención}}
	
	\vspace{20pt}
	
	% Logos
	\begin{columns}
		\begin{column}{0.17\textwidth}
			\begin{figure}
				\centering
				\includegraphics[width=1\textwidth]{logos/mm.pdf}
			\end{figure}
		\end{column}
		\begin{column}{0.5\textwidth}
			\begin{figure}
				\centering
				\includegraphics[width=1\textwidth]{logos/logougr.png}
			\end{figure}
		\end{column}
	\end{columns}

	\bigskip
	
		\vspace{20pt}
		
	\centering
	\Large
	Daniel Redondo Sánchez
	
	\bigskip
	

	\normalsize{
		\begin{columns}
			\begin{column}{1\textwidth}
				\centering
				{\color{blue}{\href{https://danielredondo.com/}{danielredondo.com}}}\\
				\vspace{5pt}
				{\color{blue}{\href{mailto:daniel.redondo.easp@juntadeandalucia.es}{daniel.redondo.easp@juntadeandalucia.es}}}
			\end{column}
			%\begin{column}{0.4\textwidth}
			%	\centering
			%	{\color{blue}{\href{https://twitter.com/dredondosanchez}{@dredondosanchez}}}\\
			%\end{column}
		\end{columns}
		
	}

\end{frame}
%------------------------------------------------
\appendix
\begin{frame}

\begin{figure}
	\centering
	\includegraphics[width=0.7\textwidth]{Figuras/Figura6}
\end{figure}

\end{frame}
%------------------------------------------------

\begin{frame}

\begin{figure}
	\centering
	\includegraphics[width=0.7\textwidth]{Figuras/Figura7}
\end{figure}

\end{frame}

%------------------------------------------------

\begin{frame}

	\begin{figure}
		\centering
		\includegraphics[width=0.5\textwidth]{Figuras/Figura12.pdf}
	\end{figure}

\end{frame}

%------------------------------------------------

\begin{frame}

\begin{figure}
	\centering
	\includegraphics[width=0.5\textwidth]{Figuras/Figura13.pdf}
\end{figure}

\end{frame}
%------------------------------------------------

\begin{frame}

\begin{figure}
	\centering
	\includegraphics[width=0.5\textwidth]{Figuras/art1.png}
\end{figure}

\end{frame}
%------------------------------------------------

\begin{frame}

\begin{figure}
	\centering
	\includegraphics[width=0.5\textwidth]{Figuras/art2.png}
\end{figure}

\end{frame}


%--------------------------------------------------

\begin{frame}\frametitle{Cáncer de próstata. Casos observados 1985-2013}
\vspace{-10pt}
\begin{figure}
\centering
\includegraphics[width=0.85\textwidth]{Figuras/prostata}
\end{figure}

\end{frame}

%--------------------------------------------------

\begin{frame}\frametitle{Cáncer de próstata. Casos observados 1985-2008}
\vspace{-10pt}

\begin{figure}
	\centering
	\includegraphics[width=0.85\textwidth]{Figuras/prostata_menos5}
\end{figure}

\end{frame}


%----------------------------------------------------------------------------------------

\end{document} 